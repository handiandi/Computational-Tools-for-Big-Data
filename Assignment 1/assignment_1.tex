\documentclass{article}
\usepackage[utf8]{inputenc}
\usepackage{amsmath,amssymb,graphicx}
\usepackage{cleveref}
%\usepackage{sectsty}


% Default fixed font does not support bold face
\DeclareFixedFont{\ttb}{T1}{txtt}{bx}{n}{12} % for bold
\DeclareFixedFont{\ttm}{T1}{txtt}{m}{n}{12}  % for normal

% Custom colors
\usepackage{color}

\definecolor{deepblue}{rgb}{0,0,0.5}
\definecolor{deepred}{rgb}{0.6,0,0}
\definecolor{deepgreen}{rgb}{0,0.5,0}
\usepackage{listings}

%\usepackage{xcolor}

% \lstdefinestyle{python}{%
%     language=python,
%     breaklines=true,
%     tabsize=4,
%     basicstyle=\ttfamily\small,
%     otherkeywords={1, 2, 3, 4, 5, 6, 7, 8 ,9 , 0, -, =, +, [, ], (, ), \{, \}, :, *, !},
%     keywordstyle=\color{blue},
%     stringstyle=\color{red},
%     showstringspaces=false,
%     emph={class, pass, in, for, while, if, is, elif, else, not, and, or, OR
%     def, print, exec, break, continue, return},
%     emphstyle=\color{black}\bfseries,
%     emph={[2]True, False, None, self},
%     emphstyle=[2]\color{key},
%     emph={[3]from, import, as},
%     emphstyle=[3]\color{blue},
%     morecomment=[s]{"""}{"""},
%     commentstyle=\color[GRAY]{0.4}\slshape,
%     rulesepcolor=\color{blue},
% }

% Python style for highlighting
\newcommand\pythonstyle{\lstset{
language=Python,
basicstyle=\ttm,
otherkeywords={self},             % Add keywords here
keywordstyle=\ttb\color{deepblue},
emph={MyClass,__init__},          % Custom highlighting
emphstyle=\ttb\color{deepred},    % Custom highlighting style
stringstyle=\color{deepgreen},
frame=tb,                         % Any extra options here
showstringspaces=false            % 
}}

% Python environment
\lstnewenvironment{python}[1][]
{
\pythonstyle
\lstset{#1}
}
{}
% Python for external files
\newcommand\pythonexternal[2][]{{
\pythonstyle
\lstinputlisting[#1]{#2}}}

\title{Assignment 1\\02807 Computational Tools for Big Data}
\author{Anonymous Authors}
\date{28th September 2015}

\begin{document}

\maketitle

\section{Exercise 1.1}
Write a command that finds the 10 most popular words in a file.
%\lstinputlisting[language=bash]{"../Lesson 1/exercise1-1.sh"}

\section{Exercise 1.2}
Put this data (https://www.dropbox.com/s/d5c4x905w4jelbu/cars.txt?dl=0) into a file and write a command that removes all rows where the price is more than 10,000\$.
%\lstinputlisting[language=bash]{"../Lesson 1/exercise1-2.sh"}
%\lstinputlisting[language=bash]{"../Lesson 1/exercise1-2.sh"}





\section{Exercise 1.3}
Using this file (https://www.dropbox.com/s/85fbd4l8s52f7to/dict?dl=0) as a dictionary, write a simple spellchecker that takes input from stdin or a file and outputs a list of words not in the dictionary. One solution gets 721 misspelled words in this Shakespeare file (https://www.dropbox.com/s/bnku7grfycm8ii6/shakespeare.txt?dl=0).
%\lstinputlisting[language=bash]{"../Lesson 1/exercise1-3.sh"}



\section{Exercise 1.4}
Launch a t2.micro instance on Amazon EC2. Log onto the instance, create some files and install some software (for example git).
\section{Exercise 1.5}
Create a few files locally on your computer. Create a new repository on Github and push your files to this repository. Log on to a t2.micro instance on Amazon EC2 and clone your repository there. Make some changes to the files, push them again and pull the changes on your local machine.

\section{Exercise 2.1}
And you can even put Python code in here!

\pythonexternal{"../Lesson 2/exercise2-1.py"}

% \section{Exercise 2.2}
% \lstinputlisting[style=python]{"../Lesson 2/exercise2-2.py"}

% \section{Exercise 2.3}
% \lstinputlisting[style=python]{"../Lesson 2/exercise2-3.py"}

% \section{Exercise 3.1}
% \lstinputlisting[style=python]{"../Lesson 3/exercise3-1.py"}
% \section{Exercise 3.2}
% \lstinputlisting[style=python]{"../Lesson 3/exercise3-2.py"}
% \section{Exercise 3.3}
% \lstinputlisting[style=python]{"../Lesson 3/exercise3-3.py"}
% \section{Exercise 3.4}
% \lstinputlisting[style=python]{"../Lesson 3/exercise3-4.py"}
% \section{Exercise 3.5}
% \lstinputlisting[style=python]{"../Lesson 3/exercise3_5_python.py"}
% \lstinputlisting[style=python]{"../Lesson 3/exercise3_5_cython.pyx"}
% \section{Exercise 4.1}
% \lstinputlisting[style=python]{"../Lesson 4/run_dbscan_cython.py"}
% \lstinputlisting[style=python]{"../Lesson 4/dbscan_cython.pyx"}


%------------

% Write a script with two methods. The first method should read in a matrix like the one here and return a list of lists. The second method should do the inverse, namely take, as input, a list of lists and save it in a file with same format as the initial file. The first method should take the file name as a parameter. The second method should take two arguments, the list of lists, and a filename of where to save the output.
% %\lstinputlisting[style=python]{"../Lesson 2/exercise2-1.py"}

% \section{Exercise 2.2}
% Write a script that takes an integer N, and outputs all bit-strings of length N as lists. For example: 3 -> [0,0,0], [0,0,1],[0,1,0],[0,1,1],[1,0,0],[1,0,1],[1,1,0],[1,1,1]. As a sanity check, remember that there are 2^N such lists. Do not use the bin-function in Python.
% %\lstinputlisting[style=python]{"../Lesson 2/exercise2-2.py"}

% \section{Exercise 2.3}
% Write a script that takes this file (from this Kaggle competition), extracts the request_text field from each dictionary in the list, and construct a bag of words representation of the string (string to count-list).

% There should be one row pr. text. The matrix should be N x M where N is the number of texts and M is the number of distinct words in all the texts.
% %\lstinputlisting[style=python]{"../Lesson 2/exercise2-3.py"}

% \section{Exercise 3.1}
% Write a script which reads a matrix from a file like this one and solves the linear matrix equation Ax=b where b is the last column of the input-matrix and A is the other columns. It is okay to use the solve()-function from numpy.linalg.
% %\lstinputlisting[style=python]{"../Lesson 3/exercise3-1.py"}

% \section{Exercise 3.2}
% Write a script that reads in this list of points (x,y), fits/interpolates them with a polynomial of degree 3. Solve for the (real) roots of the polynomial numerically using Scipy’s optimization functions (not the root function in Numpy).
% %\lstinputlisting[style=python]{"../Lesson 3/exercise3-2.py"}

% \section{Exercise 3.3}
% Do the first two exercises (Todo’s) at the bottom of http://byumcl.bitbucket.org/bootcamp2013/labs/pandas.html
% %\lstinputlisting[style=python]{"../Lesson 3/exercise3-3.py"}

% \section{Exercise 3.4}
% Last week you read in a dataset for this Kaggle competition and created a bag-of-words representation on the review strings. Train a logistic regression classifier for the competition using your bag-of-words features (and possibly some of the others) to predict the variable “requester_received_pizza”. For this exercise, you might want to work a little bit more on your code from last week. Use 90\% of the data as training data and 10\% as test data.

% How good is your classifier? Discuss the performance of the classifier.
% %\lstinputlisting[style=python]{"../Lesson 3/exercise3-4.py"}

% \section{Exercise 3.5}
% Write a simple Python function for computing the sum \frac{1}{1^2} + \frac{1}{2^2} + \frac{1}{3^2} + \ldots with 10,000 terms (this should be around 1.644), 500 times in a row (to make the execution time measurable). Now compile the code with Cython and see how much speedup you can achieve by this.
% %\lstinputlisting[style=python]{"../Lesson 3/exercise3_5_python.py"}
% %\lstinputlisting[style=python]{"../Lesson 3/exercise3_5_cython.pyx"}

\section{Exercise 4.1}
Implement the DBSCAN clustering algorithm to work with Jaccard-distance as its metric. It should be able to handle sparse data.
%\lstinputlisting[style=python]{"../Lesson 4/run_dbscan_cython.py"}
%\lstinputlisting[style=python]{"../Lesson 4/dbscan_cython.pyx"}


\end{document}